%% Setup to use an index card
\documentclass[10pt]{book}
\usepackage[vcentering,dvips]{geometry}
\geometry{papersize={3in,5in},total={2.9in,4.9in}}

%% render a frame marking the margins of a document
% \usepackage{showframe}

%% show landscape view
\usepackage{pdflscape}

%% Use for testing by filling junk (lorem-ipsum) information
\usepackage{lipsum}

\begin{document}

\begin{landscape}
\section*{Student submitting form 1 (RPI)}
As a student,\\
I want to submit an RPI form,\\
So that I can request to withhold/release my academic information.\\
\\
Specific: Student requesting a certain form.\\
Measurable: Can submit or cannot submit\\
Achievable: Form creation and submission is a core part of HTML.\\
Relevant (5 Why's)\\
    \indent
    UH has the form available for students to submit.\\
    \indent
    It is required by law for universities to have it.\\
    \indent
    It ensures students can control their privacy and own their own information.\\
    \indent
    According to modern law and beliefs, privacy is a birthright.\\
    \indent
    People don't like being unwillingly publicized.\\
Time-bound: Possible within one sprint
\end{landscape}

\begin{landscape}
\section*{Student submitting form 2 (Early withdrawal)}
As an international student,\\
I want to submit an Early Withdrawal form,\\
So that I can leave the UH system and/or the US.\\
\\
Specific: Student requesting a certain form.\\
Measurable: Can submit or cannot submit\\
Achievable: Form creation and submission is a core part of HTML.\\
Relevant (5 Why's)\\
    \indent
    An international student may need to fill out an Early Withdrawal form.\\
    \indent
    They may need to leave the US during their academic career.\\
    \indent
    The form allows students to leave without harming their academic status.\\
    \indent
    Not every student should be expected to complete their degree.\\
    \indent
    Medical leave, financial reasons, or family emergencies may force their hand.\\
Time-bound: Possible within one sprint
\end{landscape}

\begin{landscape}
\section*{Student editing submission}
As a student who previously submitted a form which has not been approved,\\
I want to edit my existing submission,\\
So that I can submit a form with the correct information.\\
\\
Specific: A feature that is only available under certain conditions.\\
Measurable: Can edit or cannot edit\\
Achievable: Submitting an edited form is just form creation with extra processing.\\
Relevant (5 Why's)\\
    \indent
    A student's original submission could have been returned.\\
    \indent
    It could contain errors or missing information.\\
    \indent
    All submissions cannot be expected to be perfect on the first try.\\
    \indent
    Submitters may have mistyped or input inaccurate information.\\
    \indent
    We are all only human.\\
Time-bound: Possible within one sprint
\end{landscape}

\begin{landscape}
\section*{Student viewing submission status}
As a student who previously submitted a form,\\
I want to view the status of my submission,\\
So that I can know whether it has been approved or not.\\
\\
Specific: A certain type of student in a particular situation.\\
Measurable: Can see status or cannot see status\\
Achievable: Already would be viewable from the reviewer process\\
Relevant (5 Why's)\\
    \indent
    A student must know if their form was approved.\\
    \indent
    If their form was returned or otherwise not approved, they cannot continue.\\
    \indent
    A returned form contains errors that prevent processing.\\
    \indent
    Errors in a form may result in the form not actually being useful to the submitter.\\
    \indent
    Typos or missing information may mean the form is not legally valid.\\
Time-bound: Easily possible if done after reviewer flow implementation.
\end{landscape}

\begin{landscape}
\section*{Reviewer viewing pending submissions}
As a reviewer,\\
I want to be able to see all pending submissions in a table,\\
So that I can approve or return them.\\
\\
Specific: A selected role and a defined view.\\
Measurable: Can see pending submissions or not\\
Achievable: Submissions are already stored - would just need to display.\\
Relevant (5 Why's)\\
    \indent
    A reviewer needs to see pending submissions.\\
    \indent
    So they can do their job\\
    \indent
    So forms can be properly processed\\
    \indent
    So submitters can have their forms approved\\
    \indent
    So students can accomplish what they desired from submitting their forms.\\
Time-bound: Quickly possible after implementation of submission systems.
\end{landscape}

\begin{landscape}
\section*{Reviewer approving or returning}
As a reviewer,\\
I want to approve or return pending forms,\\
So that students can fulfill their submission or correct any errors.\\
\\
Specific: A certain role performing certain tasks.\\
Measurable: Can approve/return or not\\
Achievable: Can reuse actions taken during submission\\
Relevant (5 Why's)\\
    \indent
    Reviewers must be able to review forms.\\
    \indent
    Reviewed forms can be approved or returned\\
    \indent
    Unreviewed forms block students from doing what they need to do.\\
    \indent
    Universities must eventually approve submitted forms.\\
    \indent
    It is legally required.\\
Time-bound: Possible within one sprint
\end{landscape}

\begin{landscape}
\section*{Reviewer leaving notes}
As a reviewer returning a form,\\
I want to send a note to the submitter,\\
So that they can know why the form was returned and what they should fix.\\
\\
Specific: A certain role performing a certain task.\\
Measurable: Can add a note or cannot add a note\\
Achievable: Extension of reviewer approval/return flow\\
Relevant (5 Why's)\\
    \indent
    A submitter should know what needs to be fixed.\\
    \indent
    If they don't know, they cannot fix it.\\
    \indent
    The submitter might not know what is wrong.\\
    \indent
    They might have misunderstood, misread, or skipped a question.\\
    \indent
    People make mistakes.\\
Time-bound: Easily possible after implementation of form approval/return.
\end{landscape}

\end{document}